\documentclass[12pt,a4paper]{article}
\usepackage{latexsym,a4,times}
\usepackage[french]{babel}
\usepackage[T1]{fontenc}
\usepackage[ansinew]{inputenc}
\setlength{\oddsidemargin}{-10mm}
\setlength{\evensidemargin}{2mm}
\setlength{\textwidth}{170mm}
\setlength{\textheight}{225mm}
\setlength{\topmargin}{0mm}
\setlength{\parskip}{1mm}

\title{Cours de syst�me d'exploitation}
\author{Christophe Gouinaud}
\begin{document}

\thispagestyle{empty}

\centerline{\Large \bf TP prog sys num�ro 5 : memory maping et verrouillage simple}
\centerline{\small {\bf ISIMA} ZZ 2 }

\medskip\medskip\medskip

\section{Mappons un r�sultat}


\begin{itemize}

\item Reprogrammer l'exercice de floutage d'image en trois processus g�rant respectivement  le rouge, le vert et le bleu en utilisant du memory mapping. Vous n'utiliserez qu'un tableau de donn�e en memory mapping en faisant en sorte de ne pas boussiller l'image d'origine. Pour cela on utilisera la fonction truncate pour cr�er le fichier.

\item Calculer la moyenne du temps n�cessaire a cr�er un pixel pour chaque couche et comparer le a la technique utilisant un pipe.

\item Calculer le taux d'asynchronisme (distance entre les pixels trait�s par les trois processus). Que remarque-t-on ?

\end{itemize}

\medskip\medskip

\section{Verouillage de fichier}

\begin{itemize}

\item Reprendre le programme permettant de discuter a plusieur en utilisant un fichier commun.  Ecrire un programme qui ecrit de fa�on aleatoire dans le canal de communication ( voir drand48() ) un message fixe identifiable. Le fichiers de discution doit normalement se corompre au bout d'un certain temps.

\item Am�liorer ce programme pour qu'il evite que deux processus utilisateurs �crivent dans le fichier de conversation en meme temps enn utilisant les fonction de verouiilage de fichier (flock ...)

\item Essayer le programme en le faisant tourner en meme temps sur ratus et etud. que se passe t-il.

\end{itemize}


\end{document}

